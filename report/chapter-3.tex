\chapter{Implementation}

\section{Preprocessing}
The first step, before inserting the data into the tables, would be to create 
copy of (a part of) the wifilog datasource, and make it part of the SOS datamodel.
Each record in the raw wifilog datasource can be seen as an observation of a device 
by an access point at a certain location. The building of which the access point is 
located in is known, but the exact location in the building is not known. For this 
reason the access points in a building are snapped to one point (approximately the 
center of the building). This has as a result that for instance, all access points 
in the Faculty of Architecture will have the same location. The table containing each 
access point, including the geometry. The geometrical location of each building on the 
campus and the raw wifilog datasource are used to create the access\_points table. 
This table contains each and every access points including the geometry. The location 
of the access points are required for the sensor observation service. 

%maybe we can put a screenshot of the access_points table here (access_points_table.png in Drive)

\section{Filling the tables}
The data is inserted into the tables using a combination of Python and SQL scripts 
(see code document). This subsection will describes our implementation of the wifilog 
datasource into the 52North SOS datamodel. %degrees symbol in 52North

%code space table
The codeSpace (see \autoref{codespace}) is the attribute for the identifier (gml:identifier) of the access points. As the GML 3.2. schema defines the gml:identifier is a “special identifier assigned to an object by the maintaining authority with the intention that it is used in references to the object.” The attribute codeSpace is of type anyURI. In the present case the Technical University Delft is the maintaining authority of the campus WLAN and with it the access points. Because there is no official repository that contains the identifiers of the access points, we defined the codespace as tudelft-wlan.

\begin{table}[]
\centering
\caption{codeSpace table}
\label{codespace}
\begin{tabular}{|l|l|}
\hline
codespaceid & 1\\ \hline
codespace & tudelft-wlan\\ \hline
\hline
\end{tabular}
\end{table}

%featureofinterest table
Both the identifier and name are corresponding to the name of the measurement station. In our case this is the name of the access point (apname) as it is stated in the wifilog database.

\begin{table}[]
\centering
\caption{FeatureOfInterest table}
\label{FOI}
\begin{tabular}{@{}ll@{}}
featureOfInterestId     & id of the featureOfInterest                                                                                                            \\
hibernatediscriminator  & F                                                                                                                                      \\
featureofinteresttypeid & 1 (there is only one type of featureofinterest)                                                                                        \\
identifier              & the name of the measurement station (the Wi-Fi access point, i.e. A-12-0-005)                                                          \\
codespace               & 1; the (only) entry in the codespace table                                                                                             \\
name                    & the name of the measurement station (the Wi-Fi access point, i.e. A-12-0-005)                                                          \\
codespaceName           & 1; the (only) entry in the codespace table                                                                                             \\
description             & \begin{tabular}[c]{@{}l@{}}the maploc i.e. System Campus;  the textual description of this feature \\ 23-CITG Beganegrond\end{tabular} \\
geom                    & the geometry of the station                                                                                                            \\
descriptionXml          &                                                                                                                                        \\
url                     &                                                                                                                                       
\end{tabular}
\end{table}

%featureOfInterestType table
In our case the featureOfInterest (see \autoref{FOIT}) is a Wi-Fi access point, with their concrete locations of measurements represented through sampling points (the featureofinteresttype).

\begin{table}[]
\centering
\caption{FeatureOfInterestType table}
\label{FOIT}
\begin{tabular}{@{}ll@{}}
\toprule
featureOfInterestTypeId & 1; id of the featureOfInterestType                                                                                                                  \\ \midrule
featureOfInterestType   & \begin{tabular}[c]{@{}l@{}}http://www.opengis.net/def/samplingFeatureType/OGC-OM/2.0/\\ SF\_SamplingPoint; featureOfInterestType value\end{tabular} \\ \bottomrule
\end{tabular}
\end{table}

%observableProperty table
We could distinguish four different observable properties (phenomena, see \autoref{obsprop}); mac, netid, rssi and snr. In our implementation the mac address, which is a unique identifier assigned to the device, is used, since complex values are only supported for SOS version 4.4.

\begin{table}[]
\centering
\caption{ObservableProperty}
\label{obsprop}
\begin{tabular}{@{}ll@{}}
\toprule
\textbf{observablepropertyid} & \textbf{1; the id of the observableProperty (which is the mac address)} \\ \midrule
hibernatediscriminator        & F                                                                       \\ \midrule
identifier                    & mac\_address; the identifier of the observable property (phenomenon)    \\ \midrule
codespace                     & 1; the (only) entry in the codespace table                              \\ \midrule
name                          & Mac; the name of the observable property                                \\ \midrule
codespacename                 & 1; the (only) entry in the codespace table                              \\ \midrule
description                   & Mac addresses scanned by wifi access point; long name of the phenomenon \\
disabled                      & F                                                                       \\ \bottomrule
\end{tabular}
\end{table}

%series table
\autoref{series} describes a serie, a combination of a featureOfInterest, observableProperty and procedure. The number of series is equal to the number sensors, the Wi-Fi access points. This means that every  feature of interest contains one single unique serie.

\begin{table}[]
\centering
\caption{Series table}
\label{series}
\begin{tabular}{@{}ll@{}}
\toprule
\textbf{seriesid}    & \textbf{id of the series}                                                       \\ \midrule
featureofinterestid  & pointing to the entry in the featureOfInterest table                            \\ \midrule
observablepropertyid & 1; pointing to the first entry in the observableProperty table (the mac adress) \\ \midrule
procedureid          & 1; the (only) entry in the procedure table                                      \\ \midrule
deleted              & F                                                                               \\ \midrule
published            &                                                                                 \\ \midrule
firsttimestamp       &                                                                                 \\ \midrule
lasttimestamp        &                                                                                 \\ \midrule
firstnumericvalue    &                                                                                 \\ \midrule
lastnumericvalue     &                                                                                 \\
unitid               &                                                                                 \\ \bottomrule
\end{tabular}
\end{table}

%observation table
\begin{table}[]
\centering
\caption{My caption}
\label{my-label}
\begin{tabular}{@{}ll@{}}
\toprule
observationid & id of the observation\\ \midrule
seriesid & pointing to the entry in the series table\\ \midrule
phenomenontimestart & asstime i.e. 2016-04-04 16:45:13; timestamp for which the measured value applied\\ \midrule
phenomenonTimeEnd & asstime+sesdur i.e. 2016-04-04 17:52:00; end of the measurement\\ \midrule
resulttime & asstime+sesdur i.e. 2016-04-04 17:52:00; end of the measurement\\ \midrule
identifier &\\ \midrule
codespace & 1; the (only) entry in the codespace table\\ \midrule
name & Scan; the name of the observation\\ \midrule
codespacename &\\ \midrule
description & WiFi Scan; the textual description of this observation\\ \midrule
deleted &\\ \midrule
validtimestart &\\ \midrule
validtimeend &\\ \midrule
unitid & 1; the (only) entry in the unit table\\ \midrule
samplinggeometry & the geometry of the station\\ \bottomrule
\end{tabular}
\end{table}


%observationhasoffering table

\begin{table}[]
\centering
\caption{ObservationHasOffering}
\label{my-label}
\begin{tabular}{@{}ll@{}}
\toprule
\textbf{observationid} & \textbf{id of the observation}            \\ \midrule
offeringid             & 1; the (only) entry in the offering table \\ \bottomrule
\end{tabular}
\end{table}

%procedure table

\begin{table}[]
\centering
\caption{My caption}
\label{my-label}
\begin{tabular}{@{}ll@{}}
\toprule
procedureid & 1; the (only) id of the procedure\\ \midrule
hibernatediscriminator & F\\ \hline
proceduredescriptionformatid & 1; the (only) entry in the proceduredescriptionformatid table\\ \midrule
identifier & wifi\_access\_point, the identifier of the procedure\\ \midrule
codespace & 1; the (only) entry in the codespace table\\ \midrule
name & Scan; the name of the procedure\\ \midrule
codespacename & 1; the (only) entry in the codespace table\\ \midrule
description & Scan; the name of the observation\\ \midrule
deleted & F\\ \midrule
disabled & F\\ \midrule
descriptionfile &\\ \midrule
referenceflag & F\\ \bottomrule
\end{tabular}
\end{table}


%proceduredescriptionformat table

\begin{table}[]
\centering
\caption{My caption}
\label{my-label}
\begin{tabular}{@{}ll@{}}
proceduredescriptionformatid & id of the procedureDescriptionFormat                                    \\
proceduredescriptionformat   & http://www.opengis.net/sensorML/1.0.1; procedureDescriptionFormat value
\end{tabular}
\end{table}

%unit table

\begin{table}[]
\centering
\caption{My caption}
\label{my-label}
\begin{tabular}{@{}ll@{}}
\toprule
\textbf{unitid} & \textbf{1; id of the unit}                                \\ \midrule
unitid          & mac adress; unit value representing a unit of measurement \\ \bottomrule
\end{tabular}
\end{table}

%textvalue table

\begin{table}[]
\centering
\caption{My caption}
\label{my-label}
\begin{tabular}{@{}ll@{}}
\toprule
\textbf{observationid} & \textbf{id of the observation, pointing to the entry in the observation table}           \\ \midrule
value                  & the measured value i.e. rhQO/vg4AnEVjXNqgQscJu8Q3bXC+JiBL5e4FR8ZxPE= (hashed mac adress)
\end{tabular}
\end{table}