\chapter{Conclusions and recommendations}
\textbf{Conclusions}\\
From current status of the application, it can be concluded that to some extent the 52°North web app is suitable for publishing Wi-Fi tracking data. However, not every part of the implementation currently works as it should. Creating a database compliant with the SOS database model is very feasible, if one takes time to understand the database model and has understanding on how to populate the tables in the database. As discussed in \autoref{H2}, mapping the correct attribute values to the correct tables is very important and should be done carefully. \\

\textbf{Recommendations}
For this project research into the implementation of Sensor Observation Services for Wi-Fi tracking data was conducted, but other services were disregarded. There was discussion about whether or not WFS would be a better service to publish Wi-Fi tracking data, but no in depth research was conducted. For better assessment, WFS should be investigated.\\
Furthermore, this project focusses SOS only, but with SOS as starting point, a Sensor Event Service (SES) could serve as a standard to push notifications to users when there is new data available. In further research, this could be a useful addition when creating an application to publish and monitor Wi-Fi tracking data. \\
Additionally,  
