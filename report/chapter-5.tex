\chapter{Conclusions and recommendations}
\textbf{Conclusions}\\
From current status of the application, it can be concluded that to some extent 
the 52°North web app is suitable for publishing Wi-Fi tracking data. However, 
not every part of the implementation currently works as we expected. Creating a 
database compliant with the SOS database model is very feasible, if one takes 
time to understand the database model and has understanding on how to populate 
the tables in the database. As discussed in \autoref{H2}, mapping the correct 
attribute values to the correct tables is very important and should be done 
carefully.\\
The company Geonovum created their own SOS pilot project (see \href{http://sospilot.readthedocs.io/en/latest/}{website}) 
and they are using SOS Transactions (SOS-T) to populate the tables in the 
database. In their opinion populating the tables directly is cumbersome and 
error prone and we can confrim that. Therefore we also recommend to use 
SOS-T in future applications.

\textbf{Recommendations}\\
For this project research into the implementation of Sensor Observation 
Services for Wi-Fi tracking data was conducted, but other services were 
disregarded. There was discussion about whether or not WFS would be a better 
service to publish Wi-Fi tracking data, but no in depth research was conducted. 
For better assessment, WFS should be investigated.\\
Furthermore, this project focusses on SOS only, but with SOS as starting point, 
a Sensor Event Service (SES) could serve as a standard to push notifications to 
users when there is new data available. The SES could open up applications in 
the field of real-time occupation monitoring for facility management 
organizations.\\
Visualizing the observation results would provide additional insight into the 
observed phenomenon. In the case of Wi-Fi tracking data, aggregated 
measurements, e.g. number of people over time are often of more value than the 
raw measurements. Therefore future research could investigate the posibilites 
to publish and visualize aggregated measurements, eventually to provide the 
option for the user to switch between the two.


