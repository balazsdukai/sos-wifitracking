\chapter{Functionality}
% General introduction, what will the chapter discuss?
The following chapter describes the OGC SOS standard implementaion by the
52°North organization. The OGC SOS 2.0 standard was adopted in 2012 which
serves the basis of the 52°North SOS 4.x implementation. Firstly the
functionalities of the implementation are detailed, then the underlying data model is described.

\section{52North application}
% General introduction of the 52N application, what can it do, implementation, services
The 52°North SOS 4.x supports the requriements of the SOS 2.0 specification,
implementing all of its extensions and their operations:
\begin{enumerate}
    \item Core
        \begin{itemize}
            \item GetCapabilities, GetObservation, DescribeSensor
        \end{itemize}
    \item Enhanced
        \begin{itemize}
            \item GetFeatureOfInterest, GetObservaitonById
        \end{itemize}
    \item Transactional
        \begin{itemize}
            \item InsertSensor, UpdateSensorDescription, DeleteSensor,
            InsertObservation
        \end{itemize}
    \item Result Handling
        \begin{itemize}
            \item InsertResultTemplate, InsertResult, GetResultTemplate,
            GetResult
        \end{itemize}
\end{enumerate}

Additionally, 52°North SOS 4.x offers the following main features:
\begin{itemize}
    \item SOS API
    \item Persistence framework to easily change the underlying database
    management system and database model (\textit{Hibernate} and
    \textit{Hibernatespatial})
    \item Administration GUI
    \item Installer GUI
    \item Bundle including SWC REST-API and JavaScript SOS Client
    \item RESTFul binding extension
\end{itemize}

\subsection{Database model}
% The database model, why did we chose non-transactional and other choices / assumptions we made 

\subsection{Standards}
% Not sure if needed..

