\chapter{Functionality}
% General introduction, what will the chapter discuss?
The following chapter describes the OGC SOS standard implementaion by the
52°North organization. The OGC SOS 2.0 standard was adopted in 2012 which
serves the basis of the 52°North SOS 4.x implementation. Firstly the
functionalities of the implementation are detailed, then the underlying data model is described.

\section{52North application}
% General introduction of the 52N application, what can it do, implementation, services
The 52°North SOS 4.x supports the requriements of the SOS 2.0 specification,
implementing all of its extensions and their operations:
\begin{enumerate}
    \item \textbf{Core}: GetCapabilities, GetObservation, DescribeSensor
    \item \textbf{Enhanced}: GetFeatureOfInterest, GetObservaitonById
    \item \textbf{Transactional}: InsertSensor, UpdateSensorDescription, DeleteSensor,
            InsertObservation
    \item \textbf{Result Handling}: InsertResultTemplate, InsertResult, GetResultTemplate,
            GetResult
\end{enumerate}

Additionally, 52°North SOS 4.x offers the following main features:
\begin{itemize}
    \item SOS API
    \item Persistence framework to easily change the underlying database
    management system and database model (\textit{Hibernate} and
    \textit{Hibernatespatial})
    \item Administration GUI
    \item Installer GUI
    \item Bundle including SWC REST-API and JavaScript SOS Client
    \item RESTFul binding extension
\end{itemize}

\subsection{Database model}
% The database model, why did we chose non-transactional and other choices / assumptions we made 
The 52°North SOS 4.x database model is divided into two major profile,
\textbf{Core} and \textbf{Transactional} profile. The Core database model
implements the OGC SOS 2.0 \textit{Core} and \textit{Extended} profiles, while
the Transactional database model extends the Core model to implement the OGC SOS
2.0 \textit{Transactional} and \textit{Result Handling} profiles. 

In the Core database model, the following tables (\autoref{cheese})are relevant
for disseminating Wi-Fi tracking data.

\begin{table}[]
\centering
\caption{Tables in the Core database model}
\label{cheese}
\begin{tabular}{ll}
\hline
\textbf{Table name}                 & \textbf{Description}                                                                                                                                                                                   \\ \hline
codespace                           & \begin{tabular}[c]{@{}l@{}}contains the \textit\{codespace\} and \textit\{codespacename\} for the identifiers of\\  Procedure, ObservableProperty, FeatureOfInterest and Offering\end{tabular}         \\
\textit{featureOfInterest}          & the geometries of the observations                                                                                                                                                                     \\
\textit{featureOfInterestType}      & the type of the geometries of the observations                                                                                                                                                         \\
\textit{observableProperty}         & parameters that are observed                                                                                                                                                                           \\
\textit{series}                     & \begin{tabular}[c]{@{}l@{}}describes a series, a combination of featureOfInterest, observableProperty\\  and procedure\end{tabular}                                                                    \\
\textit{observation}                & contains the observations                                                                                                                                                                              \\
\textit{observationHasOffering}     & relates the offerings to the observations                                                                                                                                                              \\
\textit{offering}                   & needed for structuring the data in a SOS server                                                                                                                                                        \\
\textit{procedure}                  & the processes thorough which the observed values were generated                                                                                                                                        \\
\textit{procedureDescriptionFormat} & the format in which procedures shall be described                                                                                                                                                      \\
\textit{unit}                       & units of a measurement                                                                                                                                                                                 \\
\textit{textvalue}                  & contains the measured values, which are of type OM\_TextObservation                                                                                                                                    \\
\textit{observationConstellation}   & \begin{tabular}[c]{@{}l@{}}look-up table to check if the observation type of the InsertObservation, \\ request is valid for the constellation procedure, observableProperty,,and offering\end{tabular} \\
\textit{observationType}            & stores the observation types                                                                                                                                                                           \\ \hline
\end{tabular}
\end{table}

In the Transactional database model, the following tables (\autoref{wine}) are
relevant for disseminating Wi-Fi tracking data.

\begin{table}[]
\centering
\caption{Tables in the Transactional database model}
\label{wine}
\begin{tabular}{ll}
\hline
\textbf{Table name}                     & \textbf{Description}                                                                                                                    \\ \hline
\textit{featureRelation}                & hierarchy of features                                                                                                                   \\
\textit{offeringAllowedFeatureType}     & look-up table for the feature types which are valid for the offering                                                                    \\
\textit{offeringAllowedObservationType} & look-up table for the observation types which are valid for the offering                                                                \\
\textit{resultTemplate}                 & \begin{tabular}[c]{@{}l@{}}holds the result template information which are necessary for the \\ result handling operations\end{tabular} \\
\textit{validProcedureTime}             & stores the procedure description                                                                                                       
\end{tabular}
\end{table}

